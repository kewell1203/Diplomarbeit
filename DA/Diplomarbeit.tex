%        File: Diplomarbeit.tex
%     Created: Fri Jan 02 04:00 PM 2015 C
% Last Change: Mon Jan 19 12:00 AM 2015 C
%      Author: Eduard Zhu
%       Email: Kewell1203@gmail.com

\documentclass[a4paper, twoside, 11pt]{article}

%------------------------------%
\synctex=1
%----------- preamble ---------%
%----------- packages ---------%
\usepackage[body={15cm, 23cm}, top=4.5cm, left=4cm]{geometry}
\usepackage{fancyhdr}
\usepackage{amsmath}
\usepackage{amssymb}
\usepackage{amsthm}
\usepackage{mathrsfs}
\usepackage[perpage, symbol]{footmisc}
\usepackage[T1]{fontenc}
\usepackage[utf8]{inputenc}
\usepackage{enumitem}
\usepackage{mathtools}

%----------- pagestyle setting ----------%
\pagestyle{fancy}
\fancyhead{}
\fancyfoot{}
\renewcommand{\headrulewidth}{.4pt}
\renewcommand{\footrulewidth}{.4pt}
\fancyhead[RO]{\leftmark}
\fancyhead[LE]{\rightmark}
\fancyfoot[LE, RO]{\large \thepage}

%---------- new commands ---------%
\theoremstyle{definition}
\newtheorem{definition}{\scshape Definition}[section]
\newtheorem{theorem}[definition]{\scshape Theorem}
\newtheorem{lemma}[definition]{\scshape Lemma}
\newtheorem{proposition}[definition]{\scshape Proposition}
\newtheorem{corollary}[definition]{\scshape Corollary}
\newtheorem{example}[definition]{\scshape Example}
\renewcommand{\proofname}{\upshape\bfseries Proof.}
\renewcommand{\theequation}{\thesection.\arabic{equation}}


%---------- definitions of math -----------%
% R, N
\def\RR{$\mathcal{R}$}
\def\NN{$\mathcal{N}$}
\def\AA{$\mathscr{A}$\ }
% complement
\newcommand{\compl}[1]{{#1}^{c}}
% sigma algebra
\def\sa{$\sigma$- Algebra\ } 
% prob. space
\def\bs{$(\Omega, \mathscr{A}, \mathcal{P})$\ } 
\def\bsigma{\mathscr{B}\brkt{\mathbb{R}^{n}}}
\newcommand{\sqbr}[1]{\left[ {#1} \right]}
\newcommand{\brkt}[1]{\left({#1} \right)}

  %---------- global variables setting -----%
  \setlength{\parindent}{0em}
  \setlength{\parskip}{1.5ex plus .5ex minus .5ex}
  \renewcommand{\baselinestretch}{1.3}
  \setfnsymbol{wiley}
  \renewenvironment{abstract}{
	\begin{center}
		  \Large
		  \textbf{Abstract}
		  \hspace{2em}
	\end{center}				
  }{}

  %---------- beginning of document --------%
  \begin{document}

  %---------- title page -----------%
  \begin{titlepage}
	\begin{center}
		\Huge
		Technische Universit\"at Dresden \\[-.4em]
		Fachrichtung Mathematik \\[.5em]
		\Large
		Institut f\"ur Mathematische Stochastik \\[4em]
		\bfseries\huge
		 Fractional Brownian Motion and its Application in Financial Mathematics \\[3em]
		\normalfont\Large
		Diplomarbeit \\
		zur Erlangung des ersten akademischen Grades \\[.5em]
		\bfseries\Large
		Diplommathematiker\\[.5em]
	  	\bfseries\Large
		(Wirtschaftsmathematik)\\[4em]
	\end{center}
	\large
	\begin{tabular}{lllrl}
		vorgelegt von & & & & \\[1.2em]
		Name: & Zhu & \hspace{1.5cm} & Vorname: & Ke \\[.5em]
		geboren am: & 03.12.1985 & & in: & Wuhan \\[3em]
	\end{tabular}
	\newline
	\begin{tabular}{l}
		Tag der Einreichung: \hspace{.5cm} 01.03.2015 \\[.5em]
		Betreuer: \hspace{.5cm} Prof.~Dr.~rer.~nat.~Martin ~Keller-Ressel
	\end{tabular}
\end{titlepage}
\thispagestyle{empty}
\mbox{}

  \newpage

  %---------- abstract -------------%
  \thispagestyle{empty}
  \begin{abstract}
	blahblah
  \end{abstract}
\newpage

%---------- contents -------------%
\thispagestyle{empty}
\mbox{}
\newpage
\fancyhead[LO, RE]{}
\fancyfoot[LE, RO]{}
\tableofcontents
\newpage
\thispagestyle{empty}
\mbox{}
\newpage

%---------- 1. introduction -------%
\fancyhead[RO]{\leftmark}
\fancyhead[LE]{\rightmark}
\fancyfoot[LE, RO]{\large \thepage}
\setcounter{section}{0}
\setcounter{page}{1}
\section{Introduction}

\newpage

%---------- 2. section ------------%
\section{Gaussian Process and Brownian Motion }
In this section we start off the general concept of probability spaces and stochastic processes. Of this, a most important case we then discribe, is Gaussian process. It bring us to introduce the Brownian Motion as a fine example.

%---------- 2.1. subsection -------%
\subsection{Probability Space and Stochastic Process }
\begin{definition}
  Let \AA be a collection of subsets of a set $\Omega$. \AA is then a \emph{$\sigma$- Algebra} on $\Omega$ if it satisfies the following conditions:
  \begin{enumerate}[topsep=0pt, itemsep=-1ex, partopsep=1ex, parsep=1ex, label=(\roman*)]
	\item $\Omega \in $ \AA.
	\item For any set $F \in \mathscr{A}$, its complement $\compl{F} \in$ \AA.
	\item If a serie $\{F_n\}_{n \in \mathbb{N}} \subseteq \mathscr{A}$, then $\cup_{n \in \mathbb{N}}F_n \in $ \AA.
  \end{enumerate}
\end{definition}

\begin{definition}
  A mapping $\mathcal{P}$ is said to be a \emph{probability measure} from $\mathscr{A}$ to $\bsigma$, if $\mathcal{P}\sqbr{\sum_{n=1}^{\infty} F_n} = \sum_{n=1}^{\infty} \mathcal{P}\sqbr{F_n}$ for any $\{F_n\}_{n \in \mathbb{N}}$ disjoint in $\mathscr{A}$ satisfying $\sum_{n=1}^{\infty}F_n \in \mathscr{A}$. 
\end{definition}

\begin{definition}
  A \emph{probability space} is defined as a triple \bs of a set $\Omega$, a \sa \AA  of $\Omega$ and a measure $\mathcal{P}$ from $\mathscr{A}$ to $\bsigma$.
\end{definition}

The $\sigma$- Algebra generated of all open sets on $\mathbb{R}^{n}$ is called the \emph{Borel $\sigma$- Algebra} which we denote as usual by $\mathscr{B}\left(\mathbb{R}^{n}\right)$. Let $\mu$ be a probability measure on $\mathbb{R}^{n}$. Indeed, $\brkt{\mathbb{R}^{n}, \mathscr{B}\brkt{\mathbb{R}^{n}}, \mu}$ is a special case that probability space on $\mathbb{R}^{n}$. A function $f$ mapping from $\brkt{\mathcal{D}, \mathscr{D}, \mu}$ into $\brkt{\mathcal{E}, \mathscr{E}, \nu}$ is \emph{measurable} if its collection of the inverse image of $\mathscr{E}$ is a subset of $\mathscr{D}$. A \emph{random variable} is a $\mathbb{R}^{n}$-valued measurable function on some probability space. Let $\mathcal{P}$ represent a probability measure, recall that in probability theory, for $B \in \bsigma$ we call $\mathcal{P}\sqbr{\left\{X \in B\right\}}$ the \emph{distribution} of $X$. We write also $\mathcal{P}_X \sqbr{\cdot}$ or $\mathcal{P}\sqbr{X}$ for convenience of the notation above.

\begin{definition}
  Let $\brkt{\Omega, \mathscr{A}, \mathcal{P}}$ be a probability space. A $n$-dimensional \emph{stochastic process} $\brkt{X_t}$ is a family of random variable such that $X_t\brkt{\omega} : \Omega \longrightarrow  \mathbb{R}^{n},  \forall t \in T$, where $T$ denotes the set of Index of Time.    
\end{definition}

\begin{definition}
  A stochastic process $\brkt{X_t}_{t \in T}$ is said to be \emph{stationary}, if the joint distribution 
\[
  \mathcal{P}\sqbr{X_{t_1},\dots,X_{t_n}} = \mathcal{P}\sqbr{X_{t_1+\tau},\dots,X_{t_n+\tau}} 
\]
for $t_1, \dots, t_n$ and $t_1+\tau,\dots,t_n+\tau \in T$. 
\label{sec:stn}
\end{definition}

Remark that, definition \ref{sec:stn} means the distribution of a stationary process is independent of a shift of time.

%---------- 2.2. subsection -------%
\subsection{Normal Distribution and  Gaussian Process}
\begin{definition}[1-dimensional normal distribution]
  A $\mathbb{R}$-valued random variable $X$ is said to be \emph{standard normal distributed}, if its distribution can be discribed as
  \[
	\mathcal{P}\sqbr{X \le x} = \int_{-\infty}^{x} (2\pi)^{-\frac{1}{2}}e^{-\frac{u^2}{2}}\,\mathop{du}  
  \]
  for $x \in \mathbb{R}$.
\end{definition}

\begin{definition}
  A $\mathbb{R}$-valued random variable $X$ is said to be \emph{normal distributed} with a \emph{mean} $\mu$ and a \emph{variance} $\sigma^2$, if
\[
  (X-\mu) / \sigma
\]
is standard normal distributed.
\end{definition}

We use a notation $X \sim Y$, which means $X$ and $Y$ have the same distribution. In similar way it is denoted by $X \sim (2\pi)^{-\frac{1}{2}}e^{-\frac{x^2}{2}}\mathop{dx} $, if it is standard normal distributed. In order to identifing the behaviour of a normal distributed random variable we recall the characteristic function in probability theory, see\cite{bauer}. 

\begin{proposition}
  Let $X$ be a $\mathbb{R}$-valued standard normal distributed random variable. The characteristic function of $X$
\begin{equation}
  \Psi_X(\xi) := \int_\mathbb{R} e^{ix\xi}\mathcal{P}\sqbr{X \in \mathop{dx}} = e^{-\frac{\xi^2}{2}}
  \label{sec:cht}
\end{equation}
for $\xi \in \mathbb{R}$.
\end{proposition}
\begin{proof}
  According to the definion of characteristic function
  \begin{equation*}
	\Psi_X(\xi) = \int_\mathbb{R} (2\pi)^{-\frac{1}{2}}e^{-\frac{x^2}{2}}e^{ix\xi}\,\mathop{dx},
  \end{equation*}
take differentiating both sides of the equation by $\xi$, then
\begin{eqnarray*}
\Psi_X'(\xi) &=& \int_\mathbb{R}(2\pi)^{-\frac{1}{2}}e^{-\frac{x^2}{2}}e^{ix\xi}ix\,\mathop{dx}\\
             &=& (-i)\cdot\int_\mathbb{R} (2\pi)^{-\frac{1}{2}}(\frac{d}{dx}e^{-\frac{x^2}{2}})e^{ix\xi}\,\mathop{dx}\\
			 &\overset{part.int.}{=}& -\int_\mathbb{R}(2\pi)^{-\frac{1}{2}}e^{-\frac{x^2}{2}}e^{ix\xi}\xi\,\mathop{dx}\\
			 &=& -\xi\Psi_X(\xi).
\end{eqnarray*}
Obviously, 
$\Psi(\xi) = \Psi(0)e^{-\frac{\xi^2}{2}}$ is the solution of the partial differential equation above, and $\Psi(0)$ is equal to $1$.
\end{proof}

In particular, the characteristic function of a normal distributed random variable with a mean $\mu$ and a variance $\sigma^2$, which denoted by $\Psi_{X_{\mu,\sigma^2}}(\xi)$, is $e^{i\mu\xi-\frac{1}{2}(\sigma\xi)^2}$. To achieve this result, we just need to substitute $x$ by $(x-\mu)/\sigma$ in the calculation before. 

\begin{definition}
  Let $X$ be a $\mathbb{R}^{n}$-valued random variable. $X$ is said to be \emph{normal distributed}, if for any $d \in \mathbb{R}^{n}$ such that $d^TX$ is normal distributed on $\mathbb{R}$.
\end{definition}
%Note that, $<d,X>$ is defined as scalar product on $\mathbb{R}^{n}$ that means $\sum_{j=1}^{n}\,d_j\cdot X_i$. 
\begin{proposition}
  Let $X$ be a $\mathbb{R}^{n}$-valued normal distributed. Then there exist $m \in \mathbb{R}^{n}$ and a positive definite symmetric matrix $\Sigma \in \mathbb{R}^{n\times n}$ such that,
  \begin{equation}
	\mathrm{E}\,e^{i\xi^TX} = e^{i\xi^Tm - \frac{1}{2}\xi^T \Sigma \xi}
	\label{sec:mcf}
  \end{equation}
  For $\xi \in \mathbb{R}^{n}$. Furthermore, the density function of $X$ is
\begin{equation}
  (2\pi)^{-\frac{d}{2}}\, (\det\Sigma) ^{-\frac{1}{2}}\,e^{-\frac{1}{2}(x-m)^T\Sigma^{-1}(x-m)}\,\mathop{dx}.
  \label{sec:dsy}
\end{equation}
\end{proposition}

Remark, the equation (\ref{sec:mcf}) can also be as definition of characteristic function of a n-dimensional normal distributed random variable. I.e., any normal distributed random variable can be characterized by form of the equation (\ref{sec:mcf}).

\begin{proof}
  Since $X$ normal distributed on $\mathbb{R}^{n}$, then $\xi^T X$ is normal distributed on $\mathbb{R}$. Due to the proposition 2.8 there is
  \begin{eqnarray*}
	\mathrm{E} e^{i\xi^T X} &=& \mathrm{E} e^{i\cdot 1 \cdot \xi^T X}\\
	                        &=& e^{i\mathrm{E}\sqbr{\xi^T X} -\frac{1}{2}\mathrm{Var}\sqbr{\xi^T X}}\\
							&=& e^{i\xi^T\mathrm{E}\sqbr{X} - \frac{1}{2}\xi^T \mathrm{Var}\sqbr{X} \xi}.
  \end{eqnarray*}
  According to the uniqueness theorem of characteristic function (Satz 23.4 in \cite{bauer}), then we can deduce the density function of the equation (\ref{sec:dsy}).
\end{proof}

A normal distributed normal random variable can be characterized by its mean and variance respectively mean vector and covariance vector because of the characteristic function.

\begin{corollary}
  A linear combination of independent normal distributed random variables has normal distribution.
\end{corollary}

\begin{proof}
  In general case, we suppose $Y_1, \cdots, Y_m$ are independent random variables on $\mathbb{R}^n$, for $c_1, \cdots, c_m \in \mathbb{R}$. Let have a look at the chracteristic function of it,
  \begin{eqnarray*}
	\mathrm{E}e^{i\xi^T\sum_{j=1}^m(c_jX_j)} &\overset{independent}{=}&\prod_{j=1}^{m} \mathrm{E}e^{i\xi^T(c_jX_j)}\\
	&=& \prod_{j=1}^m \exp\brkt{i\xi^T\mathrm{E}[c_jX_j]-\frac{1}{2}\xi^T\mathrm{Var}[c_jX_j]\xi}\\
	&=&  \exp\brkt{i\xi^T\mathrm{E}[\sum_{j=1}^{m}c_jX_j]-\frac{1}{2}\xi^T\mathrm\sum_{j=1}^{m}{Var}[c_jX_j]\xi}\\
	&\overset{independent}{=}&  \exp\brkt{i\xi^T\mathrm{E}[\sum_{j=1}^{m}c_jX_j]-\frac{1}{2}\xi^T\mathrm{Var}[\sum_{j=1}^{m}c_jX_j]\xi},
  \end{eqnarray*}
  which is a form of characteristc function of normal distribution. That means $\sum_{j=1}^m c_jX_j$ is normal distributed. 
\end{proof}

\begin{example}[Bivariate Normal Distribution]
  Suppose $S_1, S_2$ are independent random variables on $\mathbb{R}$ and have standard normal distributions. $\left(
    \begin{array}{c}
      S_1 \\
      S_2
    \end{array}
  \right)$  has standard normal joint distribution since they are independent. We define
  \begin{eqnarray}
	\left(
    \begin{array}{c}
      Y_1 \\
      Y_2
    \end{array}
	\right)
	&=& 
	\left(
    \begin{array}{l}
	  \sigma_1,\hspace{3em} 0 \\
	  \sigma_2 \rho, \sigma_2(1-\rho^2)^{\frac{1}{2}}
    \end{array}
  \right) \cdot
	\left(
    \begin{array}{c}
      S_1 \\
      S_2
    \end{array}
  \right)  +
  \left(
    \begin{array}{c}
      \mu_1 \\
      \mu_2
    \end{array}
  \right)
  \label{sec:bi},
  \end{eqnarray}
  where $\mu_1, \mu_2, \sigma_1, \sigma_2 \in \mathbb{R}, -1 \le \rho \le 1$. Again, $Y_1, Y_2$ are normal distributed and the joint distribution  $\left(
    \begin{array}{c}
      Y_1 \\
      Y_2
    \end{array}
	\right)$ is normal. We set $\mathrm{E}[Y_1] = \mu_1, \mathrm{E}[Y_2] = \mu_2 $ for short. Since $S_1, S_2$ are independent,
	\begin{eqnarray*}
	  \mathrm{Var}[Y_1] &=& \mathrm{Var}[\sigma_1 S_1]\\
						 &=& \sigma_1^2 ,\\
						 \mathrm{Var}[Y_2] &=& \mathrm{Var}[\sigma_2\rho S_1] + \mathrm{Var}[\sigma_2 (1-\rho^2)^{\frac{1}{2}} S_2]\\
						 &=& \sigma_2^2 \rho^2 + \sigma_2^2(1 - \rho^2)\\
						 &=& \sigma_2^2,\\
						 \mathrm{Cov}[Y_1, Y_2] &=& \mathrm{E}[(Y_1 - \mathrm{E}[Y_1])(Y_2 - \mathrm{E}[Y_2])]\\
						 &=& \mathrm{E}[Y_1Y_2 - \mu_1Y_2 - \mu_2Y_1 + \mu_1\mu_2]\\
						 &=& \mathrm{E}[(\sigma_1 S_1 + \mu_1)(\sigma_2\rho S_1 + \sigma_2(1-\rho^2)^{\frac{1}{2}}S_2 + \mu_2)] - \mu_1\mu_2\\
						 &=& \sigma_1\sigma_2\underbrace{\mathrm{E}[S_1^2]}_{=1}\rho + \mu_1\sigma_2\rho\underbrace{\mathrm{E}[S_1]}_{=0} +
						 \sigma_1\sigma_2(1-\rho^2)^{\frac{1}{2}}\underbrace{\mathrm{E}[S_1S_2]}_{=\mathrm{E}[S_1]\mathrm{E}[S_2]=0} \\
						 &+& \mu_1\sigma_2(1-\rho^2)^{\frac{1}{2}}\underbrace{\mathrm{E}[S_2]}_{=0} + \sigma_1\underbrace{\mathrm{E}[S_1]}_{=0}\mu_2 + \mu_1\mu_2 - \mu_1\mu_2\\
						 &=& \rho\sigma_1\sigma_2,
	\end{eqnarray*}
	that means the corrlation of $Y_1, Y_2$ is $\rho$.
	Because of the equation (\ref{sec:dsy}), the joint density function
	\begin{eqnarray*}
	  f_{Y_1, Y_2}(y_1, y_2) &=& (2\pi)^{-1} (\det(\Sigma))^{-\frac{1}{2}} \exp\brkt{(y_1 - \mu_1) \Sigma^{-1} (y_2 - \mu_2)},
	\end{eqnarray*}
	where $\Sigma =\left(
    \begin{array}{l}
	  \sigma_1^2, \hspace{3em}0 \\
	  \sigma_2^2\rho^2, \sigma_2^2(1-\rho^2)
    \end{array}
  \right)
 $\\
 Indeed, 
 $$
 	\det(\Sigma) = (1-\rho^2)\sigma_1^2\sigma_2^2
 $$ and 
 $$
 \Sigma^{-1} = \frac{
   \left(
    \begin{array}{l}
	  \sigma_2^2(1-\rho^2), \hspace{1em}0 \\
	  -\sigma_2^2\rho,\hspace{3em}\sigma_1^2
    \end{array}
  \right)}
  {\displaystyle (1-\rho^2)\sigma_1^2\sigma_2^2}.
  $$
 Namely,
 \begin{equation}
   f_{Y_1, Y_2}(y_1, y_2) = \frac{1}{2\pi(1 - \rho^2)^{\frac{1}{2}}\sigma_1\sigma_2}\exp\brkt{-\frac{1}{2(1-\rho^2)}(z_1^2 - 2\rho z_1 z_2 + z_2^2)}
   \label{sec:jdt}
 \end{equation}
 where $z_1 = \frac{y_1-\mu_1}{\sigma_1}, z_2=\frac{y_2-\mu_2}{\sigma_2}$.
\end{example}

\begin{corollary}
  Let $Y_1, Y_2$ be $\mathbb{R}$-valued normal distributed random variables and $\left(
    \begin{array}{c}
      Y_1 \\
      Y_2
    \end{array}
	\right)$  has a joint normal distribution, then the conditional mean of $Y_2$ given $Y_1$
    $$
	\mathrm{E}[Y_2| Y_1=y_1] = \mathrm{E}[Y_2] + \rho (y_1 - \mathrm{E}[Y_1])\frac{\sigma_2}{\sigma1},
	$$
	and the conditional variance of $Y_2$ given $Y_2$
	$$
		\mathrm{Var}[Y_2| Y_1 = y_1] = \sigma_1^2 (1 - \rho^2).
	$$
	Where $\sigma_1, \sigma_2$ are standard deviations of $Y_1, Y_2$ and $\rho$ is the correlation of $Y_1, Y_2$.
\end{corollary}

\begin{proof}
  Recall the equation (\ref{sec:jdt}), we can specify the joint density function if $\sigma_1, \sigma_2, \rho$ are known. As result of this,
  $\left(
    \begin{array}{c}
      Y_1 \\
      Y_2
    \end{array}
	\right)$ has a form of the equation (\ref{sec:bi}).
  Suppose $S_1, S_2$ are independent standard normal distributed random variables. Now we have
  \begin{eqnarray*}
	S_1 &\sim& \frac{(Y_1 - \mathrm{E}[Y_1])}{\sigma_1} \\
	Y_2 &\sim& \sigma_2\rho S_1 + \sigma_2(1-\rho^2)^{\frac{1}{2}} S_2 + \mathrm{E}[Y_2],
  \end{eqnarray*}
  more precisely,
  $$
  Y_2 \sim \sigma_2\rho \frac{(Y_1 - \mathrm{E}[Y_1])}{\sigma_1}  + \sigma_2(1-\rho^2)^{\frac{1}{2}} S_2 + \mathrm{E}[Y_2].
  $$
  Take expectation of both sides, 
  \begin{equation*}
	\mathrm{E}[Y_2|Y_1=y_1] = \sigma_2\rho \frac{(y_1 - \mathrm{E}[Y_1])}{\sigma_1} + \mathrm{E}[Y_2].
  \end{equation*}
  Now consider
  \begin{eqnarray*}
	\mathrm{Var}[Y_2|Y_1=y_1] &=&  \mathrm{E}[(Y_2 - \mu_{Y_2|Y_1})^2|Y_1=y_1]\\
							  &=& \int_{-\infty}^{\infty}(y_2 - \mu_{Y_2|Y_1})^2f_{Y_2|Y_1}(y_2, y_1)\,\mathop{dy_2}\\
							  &=& \int_{-\infty}^{\infty}\sqbr{y_2 - \mu_2 - \frac{\rho\sigma_2}{\sigma_1}(y_1-\mu_1)}^2f_{Y_2|Y_1}(y_2, y_1)\,\mathop{dy_2},
  \end{eqnarray*}
  multiply both sides by the density function of $Y_1$ and integral it over by $y_1$, we have
\begin{eqnarray*}
 &\,&\int_{-\infty}^{\infty} \mathrm{Var}[Y_2|Y_1=y_1] f_{Y_1}(y_1) \mathop{dy_1} \\
 &=&\int_{-\infty}^{\infty}\,\int_{-\infty}^{\infty} \sqbr{y_2 - \mu_2 
	- \frac{\rho\sigma_2}{\sigma_1}(y_1-\mu_1)}^2\underbrace{f_{Y_2|Y_1}(y_2, y_1)\,f_{Y_1}(y_1)}_{f_{Y_1, Y_2}(y_1, y_2)} \mathop{dy_2}\,\mathop{dy_1}\\
	&\iff&\\
	&\,&\mathrm{Var}[Y_2|Y_1=y_1] \underbrace{\int_{-\infty}^{\infty}  f_{Y_1}(y_1)}_{1} \mathop{dy_1} \\
	&=& \mathrm{E}\sqbr{(Y_2 - \mu_2) - (\frac{\rho\sigma_2}{\sigma_1})(Y_1 - \mu_1)}^2 
\end{eqnarray*}

% ausmultipizieren
Also
\begin{eqnarray*}
  \mathrm{Var}[Y_2|Y_1=y_1] &=&\underbrace{\mathrm{E}[(Y_2 - \mu_2)^2]}_{\sigma_2^2} - 2\frac{\rho\sigma_2}{\sigma_1}\underbrace{\mathrm{E}[(Y_1 -\mu_1)(Y_2 - \mu_2)]}_{\rho\sigma_1\sigma_2}\\
  &+& \frac{\rho^2\sigma_2^2}{\sigma_1^2}\underbrace{\mathrm{E}[(Y_1-\mu_1)^2]}_{\sigma_1^2}\\
  &=& \sigma_2^2 - 2\rho^2\sigma^2 + \rho^2\sigma_2^2\\
  &=& \sigma_2^2 - \rho^2\sigma_2^2.
\end{eqnarray*} 
\end{proof}

\begin{definition}
  Let $(X_t)_{t \in T}$ be a $\mathbb{R}^{n}$-valued stochastic process. $(X_t)$ is said to be a \emph{gaussian process} if $X_{t_1},\dots, X_{t_n}$ has a joint normal distribution for any  $t_1 \dots t_n \in T$ and $n \in \mathbb{N}$. 
\end{definition}
The definition immediately shows for every $X_t$ in gaussian process has a normal distribution. Therefore the prior corollary is applicable to a gaussian process.


\subsection{Brownian Motion}
The brownian motion was first introduced by Bachelier in 1900 in his PhD thesis. We now give the common definition of it.
\begin{definition}
Let $(B_t)_{t\ge0}$ be a $\mathbb{R}^{n}$-valued stochastic process. $(B_t)$ is called \emph{brownian motion} if it satisfies the following conditions:
\begin{enumerate}[topsep=0pt, itemsep=-1ex, partopsep=1ex, parsep=1ex, label=(\roman*)]
  \item $B_0 = 0 $ a.s. .
  \item $(B_{t_1} - B_{t_0}),\dots,(B_{t_n} - B_{t_{n_1}})$ are independet for $0=t_0<t_1<\dots<t_n$ and $n \in \mathbb{N}$.
  \item $B_t - B_s \sim B_{t-s}$, for $0 \le s \le t < \infty$.
  \item $B_t - B_s \sim \mathcal{N}(0, t-s)^{\otimes n}$.
  \item $B_t$ is continuous in $t$ a.s. .
\end{enumerate}
\end{definition}
A usual saying for $(ii)$ and $(iii)$ is the brownian motion has independent, stationary increments. In (iv), $\mathrm{N}$ represent a random variable which has a normal distribution. $B_t$ is normal distributed due to (ii). It is clear that the increments of brownian motion is stationary.

\begin{proposition}
  Let $(B_t)$ be a one-dimensional brownian motion. Then the covarice of $B_m, B_n$ for $m, n \ge 0$ is $m \wedge n $.
\end{proposition}

\begin{proof}
  WLOG, we assume that $m \ge n$, then
  \begin{eqnarray*}
	\mathrm{E}[B_mB_n] &=& \mathrm{E}[(B_m - B_n)B_n] + \mathrm{E}[B_n^2]\\
	&=& \mathrm{E}[B_m - B_n]\mathrm{E}[B_n] + n\\
	&=& n .
  \end{eqnarray*}
  \label{sec:cor}
\end{proof}

\begin{proposition}
  Let $(B_t)$ be a one-dimensional brownian motion. Then $B_{cm} \sim c^{\frac{1}{2}}B_m$.
\end{proposition}

\begin{proof}
  Because $B_m$ is normal distributed for any $m > 0$, we then get
  \begin{eqnarray*}
	\mathrm{E} [e^{i\xi B_{cm}}] &=& e^{-\frac{1}{2}cm\xi^2}\\
	&=& e^{-\frac{1}{2}(c(m)^{\frac{1}{2}}\xi)^2}\\
	&=& \mathrm{E} [e^{i\xi c^{\frac{1}{2}}B_m}] .
  \end{eqnarray*}
\end{proof}

\begin{theorem}
  A one-dimensional brownian motion is a gaussian process.
\end{theorem}

\begin{proof}
  The following idea using the independence of increments to prove the claim come from \cite{shilling}.
  We choose $0=t_0<t_1<\dots<t_n$, for $n \in \mathbb{N}$. Define
  $V = (B_{t_1},\dots,B_{t_n})^T$,  $K = (B_{t_1}-B_{t_0},\dots, B_{t_n}-B_{t_{n-1}})^T$ and 
  $A = 
  \begin{pmatrix}
	1      & 0      & \cdots & 0\\
	1      & 1      & \cdots & 0\\
	\vdots & \vdots & \ddots & \vdots \\
	1      & 1      & \cdots & 1
  \end{pmatrix}
	$.
  Let us see the characteristic function of $V$,
  \begin{eqnarray*}
	\mathrm{E} [e^{i\xi^T V}] &=& \mathrm{E} [e^{i\xi^T AK}]\\ 
	&=& \mathrm{E} [e^{iA^T\xi K}]\\
	&=& \mathrm{E} [\exp(i (\xi^{(1)}+\dots+\xi^{(n)}, \xi^{(2)}+\dots+\xi^{(n)}, \cdots,\xi^{(n)})] \\
	&\cdot& (B_{t_1}-B_{t_0}, B_{t_2}-B_{t_1},\dots,B_{t_n}-B_{t_{n-1}})^T)\\
	&\overset{ind.increments}{=}& \prod_{j=1}^n \mathrm{E} [\exp(i(\xi^{(j)+\dots+\xi^{(n)}})(B_{t_j}-B_{t_{t-1}}))]\\
	&\overset{stat.increments}{=}& \prod_{j=1}^n \exp(-\frac{1}{2}(t_j - t_{j-1})(\xi^{(j)}+\dots+\xi^{(n)})^2) \\
	&=& \exp\left(-\frac{1}{2}\sum_{j=1}^n (t_j - t_{j-1})(\xi^{(j)}+\dots+\xi^{(n)})^2\right)\\
    &=& \exp\left(-\frac{1}{2}\left(\sum_{j=1}^n t_j(\xi^{(j)}+\dots+\xi^{(n)})^2 - \sum_{j=1}^n t_{j-1}(\xi^{(j)}+\dots+\xi^{(n)})^2\right)\right)\\
	&=& \exp\left(-\frac{1}{2}\left(\sum_{j=1}^{n-1} t_j((\xi^{(j)}+\dots+\xi^{(n)})^2 - (\xi^{(j+1)}+\dots+\xi^{(n)})^2) + t_n(\xi^{(n)})^2\right)\right)\\
	&=& \exp\left(-\frac{1}{2}\left(\sum_{j=1}^{n-1} t_j\xi^{(j)}(\xi^{(j)}+2\xi^{(j+1)}+\dots+2\xi^{(n)}) + t_n(\xi^{(n)})^2\right)\right)\\
	&=& \exp\left(-\frac{1}{2}\left(\sum_{j,h=1}^n(t_j\wedge t_h)\xi^{(j)}\xi^{(h)}\right)\right).
  \end{eqnarray*}
  Recall with proposition \ref{sec:cor}, $(t_j\wedge t_h)_{t,h=1,\dots,n}$ is the covariance matrix of $V$. The mean vector of it is zero, then we have been proved that the characteristic function is a form of some normal distributed random vector, i.e., $V$ is normal distributed.
\end{proof}

Shilling gave in his lecture \cite{shilling} the relationship between a one-dimensional brownian motion and a n-dimensional brownian motion.
$(B_t^{(l)})_{l=1,\dots,n}$ is brownian motion if and only if $B_t^{(l)}$ is brownian motion and all of the component are independent. Using this independence and te theorem of fubini in the characteristic function for high-dimensional brownian motion we can say a n-dimensional brownian motion is also a gaussian process.

\newpage

%-------- 3. section
\section{Regularity for Brownian Motion and It\'o Integral}
\subsection{L\'evy Modulus of Continuity}
We consider now the one-dimensional brownian motion. In this section we need some notations, which are defined as followings
\begin{equation*}
\Delta^{[0,T]} = \left\{t_1,\dots,t_n|0=t_0<\dots<t_n=T\right\} 
\end{equation*}
$$
|\Delta^{[0,T]}| = \max_{t_j \in \Delta^{[0,T]}}|t_j - t_{j-1}|
$$.

\begin{lemma}
  Let $B_t$ be a brownian motion. Then
  $$
  \sum_{t_j \in \Delta^{[0,T]}} |B_{t_j} - B_{t_{j-1}}|^2 \xmapsto[L^2(\mathcal{P})]{|\Delta^{[0,T]}|\rightarrow 0} T  
  $$
\end{lemma}

\newpage
%---------- 3. section ------------%
\section{Fractional Brownian Motion}

\newpage

%---------- 4. section ------------%
\section{Fractional Ornstein Uhlenbeck Process Model}

\newpage

%---------- 5. Anwendung in die finanzmathe bzw. in der volatility process -------%
\section{Application in Financial Mathematics}

\newpage

%---------- x. conclusion ---------%
\section{Conclusion}

\newpage

%---------- reference -------------%
\addcontentsline{toc}{section}{References}
\fancyhead[LO, RE]{}
\begin{thebibliography}{99}
	\bibitem{bauer} \textsc{Bauer,~H.} (2002). Wahrscheinlichkeitstheorie(5th. durchges. und verb. Aufl.). Berlin: W. de Gruyter
	\bibitem{shilling} (2012, October 1). Stochastic Processes. Lecture conducted from \textsc{Shilling, R.}, Dresden.

\end{thebibliography}
\newpage
\end{document}


