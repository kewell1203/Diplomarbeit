%        File: DA_Beamer.tex
%     Created: Sun Jun 26 05:00 PM 2011 C
% Last Change: Sun Jun 26 05:00 PM 2011 C
%
\documentclass[12pt]{beamer}
\usefonttheme[stillsansserifmath]{serif}
\usetheme{Madrid}
\defbeamertemplate*{frametitle}{smoothbars theme}
{%
\nointerlineskip%
\begin{beamercolorbox}[wd=\paperwidth,leftskip=.3cm,rightskip=.3cm plus1fil,vmode]{frametitle}
	\vskip.3ex
	\usebeamerfont*{frametitle}\insertframetitle%
	\vskip.3ex
\end{beamercolorbox}%
}
%\let\Tiny=\tiny
\usepackage{lmodern}
\def\F{\mathcal{F}}
\def\B{\mathcal{B}}
\def\RR{\mathbb{R}}
\def\NN{\mathbb{N}}
\def\i{\mathrm{i}}
\def\e{\mathrm{e}}
\def\d{\mathrm{d}}
\def\k{\mathrm{k}}
\def\Var{\mathrm{Var}}
\def\ind{\mathbf{1}}
\def\la{\langle}
\def\ra{\rangle}
\newtheorem{proposition}{Proposition}
\title[Diploma Thesis]{Additive Processes and Some Path Properties}
\author{Weijun Yu} 
\institute[TU Dresden]{Supervisor: Prof.~Dr.~rer.~nat.~R.~Schilling\\[2ex] Institute of Mathematical Stochastics\\ TU Dresden}
\date[July 2011]{July 26, 2011}
\begin{document}
\begin{frame}[plain]
	\titlepage
\end{frame}
\begin{frame}
	\frametitle{Contents}
	\begin{itemize}
		\item Introduction of the Problem
			\pause
		\item L\'evy Processes
			\pause
		\item Additive Processes
			\pause
			\begin{itemize}
				\item The Quasiconvex Function Method
					\pause
				\item The Semicontinuous Function Method
					\pause
				\item The Moment Method
					\pause
			\end{itemize}
		\item Application to L\'evy Processes
			\pause
		\item Some Other Applications 
	\end{itemize}
\end{frame}
\begin{frame}
	\frametitle{Introduction of the Problem}
	\begin{problem}
		Let $(X_t)_{t\ge0}$ be any additive process in $\RR^d$, there are four finite indices $\delta_i,\beta_i,i=1,2$ and a nondecreasing function $u$ such that a.s.
		\begin{equation}
			\begin{split}
				\liminf_{t\to0} u(t)^{-1/\eta}X_t^*=
				\begin{cases}	
					0, & \mathrm{if\ } \eta>\delta_1, \\
					\infty, & \mathrm{if\ } \eta<\delta_2, \\
				\end{cases} \\
				\limsup_{t\to0} u(t)^{-1/\eta}X_t^*=
				\begin{cases}	
					0, & \mathrm{if\ } \eta>\beta_2, \\
					\infty, & \mathrm{if\ } \eta<\beta_1, \\
				\end{cases}
			\end{split}
			\label{additive}
		\end{equation}
		where $X_t^* = \sup_{0\le s\le t}|X_s|$.\\[1ex]\pause An interesting question: When do the equations $\delta_1=\delta_2,\beta_1=\beta_2$ hold?
	\end{problem}
\end{frame}
\begin{frame}
	\frametitle{L\'evy Processes}
	Let $(X_t)_{t\ge0}$ be a real-valued L\'evy process with generating triplet $(B,Q,\nu)$. Then for any $r>0$,
	$$
	P(X_t^* \ge r) \le ? \mathrm{\ and\ } P(X_t^* \le r) \le ?
	$$
	\pause
	Decompose $X_t$ into two parts:
	$$
	X_t = Y_t^r + Z_t^r
	$$
	where $Z_t^r=\sum_{0\le s\le t}\Delta X_s \ind_{\{|\Delta X_s| > r\}}$. \\[1ex]
	\pause
	Furthermore, we can decompose $Y_t^r$ as
	$$
	Y_t^r = E Y_t^r + Y_t^{r,c} + Y_t^{r,d},
	$$
	where $E Y_t^r$ the ``drift'' at level $r$, $Y_t^{r,c}$ the Gaussian part and $Y_t^{r,d}$ the jump part with jumps smaller than $r$.
\end{frame}
\begin{frame}
	\frametitle{L\'evy Processes}
	\begin{equation*}
		\begin{split}
			& E \left[{\textstyle\sum_{0\le s\le t}} \ind_{\{|\Delta X_s| > r\}}\right] =\int_{|x|>r}(t\nu)(\d x) \triangleq tG(r), \\[1ex]
			& r^{-2} (\Var Y_t^{r,c} + \Var Y_t^{r,d}) = r^{-2}\left[ tQ+\int_{|x|\le r}x^2(t\nu)(\d x) \right] \triangleq tK(r),\\[1ex]
			& r^{-1} |E Y_t^r| = r^{-1}\left|tB+\int_{\RR^d}\left(\ind_{|x|\le r}(x)-\ind_{|x|\le 1}(x)\right)x(t\nu)(\d x) \right| \\
			& \phantom{r^{-1} |E Y_t^r|} \triangleq tM(r),\\[1.5ex]
			& h(r) \triangleq G(r)+K(r)+M(r),\quad r>0,\\[1ex]
		\end{split}
	\end{equation*}
	\pause
	Our expectation: $P(X_t^* \ge r)$ and $P(X_t^* \le r)$ are dominated by $th(r)$ and $\big(th(r)\big)^{-1}$, respectively.
\end{frame}
\begin{frame}
	\frametitle{L\'evy Processes}
	\begin{lemma}
		Let $(X_t)_{t\ge 0}$ be a L\'evy process in $\RR^d$ with generating triplets $(B,Q,\nu)$. Define $h$ as above. Then there is a constant $C(d)$ depending only on the dimension $d$ such that for any $t\ge0,r>0$,
		\begin{equation*}
			P(X_t^*\ge r) \le C(d)th(r),\quad P(X_t^*\le r) \le C(d)\big(th(r)\big)^{-1}.
		\end{equation*}
	\end{lemma}
\end{frame}
\begin{frame}
	\frametitle{L\'evy Processes}
	\begin{theorem}
		Let $(X_t)_{t\ge 0}$ be a L\'evy process in $\RR^d$ with generating triplets $(B,Q,\nu)$ and define $h$ as above. Then we have two finite indices
		\begin{eqnarray*}
			& & \delta \triangleq \inf\left\{ \eta\ge0\left|\liminf\nolimits_{r\to0}r^\eta h(r)=0\right. \right\},\\[1ex]
			& & \beta \triangleq \inf\left\{ \eta\ge0\left|\limsup\nolimits_{r\to0}r^\eta h(r)=0\right. \right\},
		\end{eqnarray*}
		such that a.s.
		\vspace{-2ex}
		\begin{equation*}
			\begin{split}
				& \liminf_{t\to0}t^{-1/\eta}X_t^*= 
				\begin{cases}
					0, & \mathrm{if\ } \eta>\delta, \\
					\infty, & \mathrm{if\ } \eta<\delta, \\
				\end{cases} \\
				& \limsup_{t\to0}t^{-1/\eta}X_t^*= 
				\begin{cases}
					0, & \mathrm{if\ } \eta>\beta, \\
					\infty, & \mathrm{if\ } \eta<\beta. \\
				\end{cases} \\
			\end{split}
		\end{equation*}
	\end{theorem}
\end{frame}
\begin{frame}
	\frametitle{Additive Processes}
	Let $(X_t)_{t\ge 0}$ be any real-valued additive process with generating triplets $(B_t,Q_t,\nu_t)$.
	\pause
	\begin{equation*}
		\begin{split}
			& G_t(r)=\int_{|x|>r}\nu_t(\d x),\qquad K_t(r)=r^{-2}\left[Q_t +\int_{|x|\le r}x^2\nu_t(\d x) \right],\\[1ex]
			& M_t(r)=r^{-1}\left| B_t+\int_{\RR^d}\left(\ind_{|x|\le r}(x)-\ind_{|x|\le 1}(x)\right)x\nu_t(\d x) \right|,\\[1ex]
			& M_t^*(r)=\max_{0\le s\le t} M_s(r),\\[1ex]
			& y_t(r)=G_t(r)+K_t(r)+M_t^*(r),\qquad t\ge0,r>0,\\
		\end{split}
	\end{equation*}
	\pause
	For a $d$-dim.~additive process, Let $y_t(r)=\sum_{i=1}^d y_t^{(i)}(r)$, where $y_t^{(i)}(r), i=1,\ldots,d$ are the $y$ functions of each component.
\end{frame}
\begin{frame}
	\frametitle{Additive Processes}
	\begin{lemma}
		Let $(X_t)_{t\ge0}$ be any additive process in $\RR^d$ with generating triplets $(B_t,Q_t,\nu_t)$ and $y_t(r)$ the function defined as above. Then for all $t\ge0,r>0$,
		\begin{equation*}
			P(X_t^*\ge r)\le\pi_d y_t(r),\quad P(X_t^*\le r)\le A_k(d)y_t(r)^{-k/2},\ k=1,2,\ldots,
		\end{equation*}
		where $\pi_d=aK(d),a=2^{-1}(3+\sqrt5),K(d)=2d^3,d>1,K(1)=1$ and $A_k(d)=(24\sqrt{3dk})^k$.
	\end{lemma}
	\pause
	\vspace{2ex}
	PROBLEM: The two variables $t$ and $r$ in $y_t(r)$ usually cannot be separated from each other!
\end{frame}
\begin{frame}
	\frametitle{Additive Processes}
	\begin{theorem}
		Let $(X_t)_{t\ge0}$ be any additive process in $\RR^d$, then there is a non- decreasing right-continuous $v$, $v(0)=0$ such that (\ref{additive}) holds with $u=v^{-1}$ and $\delta_1,\delta_2,\beta_1,\beta_2$ finite indices defined as \small
		\begin{equation*}
			\begin{split}
				& \delta_1=\inf\left\{ \eta>0 \left| \liminf_{r\to0}y_{v(r)}(r^{1/\eta})=0 \right. \right\},\\
				& \delta_2=\sup\left\{ \eta>0 \left| \sum y_{v(\sigma_n)}(\sigma_n^{1/\eta})^{-1}<\infty\mathrm{\ for\ some\ }\Sigma\mathrm{-sequence\ }\sigma_n\downarrow0 \right. \right\},\\
				& \beta_1=\sup\left\{ \eta>0 \left| \liminf_{r\to0}y_{v(r)}(r^{1/\eta})^{-1}=0 \right. \right\},\\
				& \beta_2=\inf\left\{ \eta>0 \left| \sum y_{v(\sigma_n)}(\sigma_n^{1/\eta})<\infty\mathrm{\ for\ some\ }\Sigma\mathrm{-sequence\ }\sigma_n\downarrow0 \right. \right\}.\\
			\end{split}
		\end{equation*}
		$(\Sigma\mathrm{-sequence\ }\sigma_n\downarrow0 :\Leftrightarrow \frac{\sigma_{n-1}}{\sigma_n}\cdot \sigma_n^\eta\to0,n\to\infty,\forall \eta>0)$
	\end{theorem}
\end{frame}
\begin{frame}[label=quasiconvex]
	\frametitle{The Quasiconvex Function Method}
	\begin{definition}
		A function $c:(0,1)\to(0,1)$ is called \emph{\textcolor{blue}{slow}} if $\lim_{r\to0}r^{\eta}/c(r)=0$ for all $\eta>0$.
	\end{definition}
	\pause
	\begin{definition}
		Let $(X_t)_{t\ge0}$ be any additive process in $\RR^d$ with corresponding $y_t(r)$. $v(t)$ is a nondecreasing, right-continuous function, $v(0)=0$. Then $y_{v(t)}(r)$ is called \emph{\textcolor{blue}{quasiconvex}} with respect to $v$, if
		\begin{equation*}
			y_{v(t_2)}(r)/y_{v(t_1)}(r)\ge c(r)(t_2/t_1)^\sigma,\qquad 0<t_1<t_2, r>0,
		\end{equation*}
		with a slow function $c(r)$ and a constant $\sigma>0$.
	\end{definition}
\end{frame}
\begin{frame}
	\frametitle{The Quasiconvex Function Method}
	\begin{theorem}
		In the last theorem, if $y_{v(t)}(r)$ is quasiconvex, then $\delta_1=\delta_2=\delta$, $\beta_1=\beta_2=\beta$, where
		\vspace{-1ex}
		\begin{eqnarray*}
			& & \delta=\inf\left\{ \eta\ge0\left|\limsup_{r\to0} r^{-\eta} n(r)=0\right. \right\}, \\[1ex]
			& & \beta=\inf\left\{ \eta\ge0\left|\liminf_{r\to0} r^{-\eta} n(r)=0\right. \right\} \\[1ex]
			& \mathrm{and} & n(r)=\inf\left\{ t>0|y_{v(t)}(r)>m \right\}, m>0 \mathrm{\ fixed}. \\
		\end{eqnarray*}
	\end{theorem}
\end{frame}
\begin{frame}
	\frametitle{The Semicontinuous Function Method}
	\begin{definition}
		Define for small $t>0$,
		\begin{equation*}
			\begin{split}
				& \overline v(t)=\inf\left\{ s>0\left|y_s(t^{1/\eta})\ge c(t)ty_b(t^{1/\eta})\mathrm{\ for\ all\ }\eta\in[\delta-\varepsilon,\delta)\right. \right\},\\
				& \underline v(t)=\sup\left\{ s>0\left|y_s(t^{1/\eta})\le c(t)^{-1}ty_b(t^{1/\eta})\mathrm{\ for\ all\ }\eta\in(\delta,\delta+\varepsilon]\right. \right\},\\
				& \overline u(t)=\inf\left\{ s>0\left|y_s(t^{1/\eta})\ge c(t)ty_b(t^{1/\eta})\mathrm{\ for\ all\ }\eta\in[\beta-\varepsilon,\beta)\right. \right\},\\
				& \underline u(t)=\sup\left\{ s>0\left|y_s(t^{1/\eta})\le c(t)^{-1}ty_b(t^{1/\eta})\mathrm{\ for\ all\ }\eta\in(\beta,\beta+\varepsilon]\right. \right\},
			\end{split}
		\end{equation*}
		where $\varepsilon$ is a small positive constant and $c(t)$ is a continuous slow function.
	\end{definition}
\end{frame}
\begin{frame}
	\frametitle{The Semicontinuous Function Method}
	\begin{theorem}
		\small
		Let $(X_t)_{t\ge0}$ be any additive process in $\RR^d$ and $\overline v,\underline v,\overline u,\underline u$ as given above. Then a.s.~it holds that
		$$
		\begin{array}{ll}
			\lim_{t\to0} t^{-1/\eta}X_{\overline{v}(t)}^*=\infty, \mathrm{if\ }\eta<\delta, & \hspace{-1.2ex} \liminf_{t\to0} t^{-1/\eta}X_{\underline{v}(t)}^*=0, \mathrm{if\ }\eta>\delta, \\[1.5ex]
			\limsup_{t\to0} t^{-1/\eta}X_{\overline{u}(t)}^*=\infty, \mathrm{if\ }\eta<\beta, & \hspace{-1.2ex} \lim_{t\to0} t^{-1/\eta}X_{\underline{u}(t)}^*=0, \mathrm{if\ }\eta>\beta, 
		\end{array}
		$$
		where the indices are defined as
		\vspace{-1ex}
		\begin{eqnarray*}
			& & \delta=\inf\left\{ \eta\ge0\left|\liminf_{r\to0}r^\eta y_b(r)=0\right. \right\},\\
			& & \beta=\inf\left\{ \eta\ge0\left|\lim_{r\to0}r^\eta y_b(r)=0\right. \right\}, \qquad b>0 \mathrm{\ fixed}.\\
		\end{eqnarray*}
	\end{theorem}
\end{frame}
\begin{frame}[label=semicontinuous]
	\frametitle{The Semicontinuous Function Method}
	\begin{proposition}
		The following conditions implies that $\overline v\le\underline v,\overline u\le\underline u$ in the last theorem 
		\begin{enumerate}
			\item[(1)] $y_s(r_2)/y_b(r_2)\le c_1(r_1)^{-1}c_2(r_2)^{-1}y_s(r_1)/y_b(r_1)$ for all $r_1<r_2$ small and $s\in(0,b)$. \\[1ex]
			\item[(2)] $y_s(r)/\dot y_s(r)-y_b(r)/\dot y_b(r)\le\log c(r)/\log r$ for all small $r$ and $s\in(0,b)$, where $\dot y_t(r)=\big(r^{-1}\int_0^r y_t(x)^{-1}\d x\big)^{-1}$. \\[1.5ex]
		\end{enumerate}
		Here $c_1(t),c_2(t),c(t)$ are slow functions, whose reciprocals $c_1(t)^{-1},c_2(t)^{-1},c(t)^{-1}$ are uniformly continuous on $(0,1]$.
	\end{proposition}
\end{frame}
\begin{frame}
	\frametitle{The Moment Method}
	\begin{definition}
		Let $(X_t)_{t\ge0}$ be any additive process in $\RR^d$. $e$ is a bounded, non- decreasing, absolutely continuous function with $e(0)=0$. Define
		$$
		a_e(t)=Ee(X_t^*),\quad H(r)=e(r)^{-1},\quad h(r)=(Ea_e(T_r))^{-1},
		$$
		where $T_r=\inf\{t>0| |X_t|>r\},r>0$.
	\end{definition}
	\pause
	\begin{lemma}
		$e,a_e(t),H(r),h(r)$ are defined as above. Let $u=a_e(t)$ and $v(t)=u^{-1}(t)$. Then
		$$
		P(X_{v(t)}^*\le r) \le t H(r), \qquad P(X_{v(t)}^*\le r) \le (th(r))^{-1}.
		$$
	\end{lemma}
\end{frame}
\begin{frame}
	\begin{theorem}
		Let $(X_t)_{t\ge 0}$ be any additive process in $\RR^d$. $e,a_e(t),H(r),h(r)$ are the same as above. Define
		\vspace{-1ex}
		\begin{equation*}
			\begin{split}
				& \delta_1=\inf\left\{ \eta\ge0\left|\liminf_{r\to0}r^\eta H(r)=0\right. \right\},\\
				& \delta_2=\inf\left\{ \eta\ge0\left|\liminf_{r\to0}r^\eta h(r)=0\right. \right\},\\
				& \beta_1=\inf\left\{ \eta\ge0\left|\lim_{r\to0}r^\eta h(r)=0\right. \right\},\\
				& \beta_2=\inf\left\{ \eta\ge0\left|\lim_{r\to0}r^\eta H(r)=0\right. \right\}.\\
			\end{split}
		\end{equation*}
		Then (\ref{additive}) holds with $u=a_e$. 
	\end{theorem}
\end{frame}
\begin{frame}
	\frametitle{The Moment Method}
	\begin{definition}
		A nondecreasing right-continuous function $\phi$ with $\phi(t)>0,t>0,\phi(0)=0$ is called \emph{\textcolor{blue}{moderate}} if there are two constants $\rho,\sigma\in (0,\infty)$ such that $\phi(t_2)/\phi(t_1)\le\rho(t_2/t_1)^\sigma$ whenever $0<t_1<t_2$.
	\end{definition}
	\pause
	\begin{proposition}
		Let $(X_t)_{t\ge0}$ be any additive process in $\RR^d$ and let $e$ be any (bounded) moderate function. Then
		$$
		h(r) \approx \left[e(r) + \int_r^1 EG_{T_r}(\lambda)e(\d \lambda)\right]^{-1}, r>0.
		$$
	\end{proposition}
\end{frame}
\begin{frame}
	\frametitle{The Moment Method}
	\begin{lemma}
		If we redefine $h(r)$ by $h(r) = \left[e(r) + \int_r^1 EG_{T_r}(\lambda)e(\d \lambda)\right]^{-1}$. Then $H(r)\le c(r)^{-1}h(r),r\in(0,1)$ is equivalent to
		\begin{equation}
			\int_r^1 EG_{T_r}(\lambda)e(\d \lambda) \le c(r)^{-1}e(r), r\in(0,1),
			\label{moment}
		\end{equation}
		where $c(r)$ is a slow function.
	\end{lemma}
\end{frame}
\begin{frame}[label=momentmethod]
	\frametitle{The Moment Method}
	\small
	\begin{theorem}
		Let $(X_t)_{t\ge0}$ be any additive process in $\RR^d$ and let $e$ be any (bounded) moderate function. $u=a_e,H,\delta_1,\delta_2,\beta_1,\beta_2$ are the same as in the last theorem, $h$ is newly defined as in the lemma above. Then (\ref{additive}) holds. In addition, If we have 
		\vspace{-0.5ex}
		$$
		EG_{T_r}(\lambda)\le c(r)^{-1}A(r)/A(\lambda),0\le r \le \lambda \le 1,
		$$ 
		\vspace{-0.5ex}
		for a moderate function $A$, then (\ref{moment}) holds with $e=A^{1/q},q>1$ and hence $\delta_1=\delta_2,\beta_1=\beta_2$.
	\end{theorem}
	\pause
	Proof. Under this condition we have 
	\begin{eqnarray*}
		\int_r^1 EG_{T_r}(\lambda) e(\d\lambda) &\le& c(r)^{-1}e(r)^q\int_r^1 e(\lambda)^{-q} e(\d\lambda)\\ 
		&=&  c(r)^{-1}e(r)^q\frac1{1-q}(e(1)^{1-q}-e(r)^{1-q})\\
		&=& c_1(r)^{-1}e(r). \hspace{4cm} \square
	\end{eqnarray*}
\end{frame}
\begin{frame}
	\frametitle{Application to L\'evy Processes}
	Let $(X_t)_{t\ge0}$ be a L\'evy process. Then $y_t(r)=th(r)$.
	\vspace{1ex}
	\pause
	\begin{itemize}
		\item \underline{The quasiconvex function method}: The quasiconvex condition \hyperlink{quasiconvex}{\beamerbutton{here}} is satisfied since from 
			\vspace{-1ex}
			$$
			y_{t_2}(r)/t_2 = h(r) = y_{t_1}(r)/t_1,0<t_1<t_2
			$$ 
			\vspace{-1ex}
			it follows 
			\vspace{-1ex}
			$$
			y_{t_2}(r)/y_{t_1}(r)=t_2/t_1,0<t_1<t_2.
			$$
			\vspace{-2.5ex}
			\pause
		\item \underline{The semicontinuous function method}: The two conditions \hyperlink{semicontinuous}{\beamerbutton{here}} are vacuously satisfied as \\[0.5ex] 
			$y_s(r_2)/y_b(r_2) = s/b = y_s(r_1)/y_b(r_1)$ \qquad and \\[0.5ex]
			$y_s(r)/\dot y_s(r) - y_b(r)/\dot y_b(r)$\\[0.8ex]
			$= r^{-1} \int_0^r h(r)/h(x)\d x - r^{-1} \int_0^r h(r)/h(x)\d x = 0 \le \log c(r) / \log r$,\\[0.5ex]
			for any slow function $c(r)\in (0,1)$ and $r<1$.
	\end{itemize}
\end{frame}
\begin{frame}
	\frametitle{Application to L\'evy Processes}
	\begin{itemize}
		\item \underline{The Moment Method}: 
			\begin{lemma}
				Let $(X_t)_{t\ge0}$ be any additive process in $\RR^d$ and let $e$ be any moderate function. Then there are two positive constants $C_1,C_2$ such that $C_1\le Ee(y_{T_r}(r))\le C_2,\quad r>0$.
			\end{lemma}
			By the foregoing lemma, we have $C_1 \le Ey_{T_r}(r)=ET_rh(r) \le C_2$, i.e.\ $ET_r\approx h(r)^{-1}$. Then $EG_{T_r}(\lambda) = ET_rG(\lambda)\le ET_rh(\lambda) \approx h(\lambda)/h(r)$. Furthermore, if $(X_t)_{t\ge0}$ is not a compound Poisson process, $h(r)$ is comparable to a moderate function. Then let $e(r)=h(r)^{-1/q},q>1$ and the condition \hyperlink{momentmethod}{\beamerbutton{here}} is satisfied.
	\end{itemize}
\end{frame}
\begin{frame}
	\frametitle{Some Other Applications}
	\begin{itemize}
			\pause
		\item Additive processes with $y_t(r) \approx f(t)z(r)$;
			\vspace{2ex}
			\pause
		\item Continuous additve processes. Then $\nu_t = 0,\forall t\ge0$ and $y_t(r) = B_t^* r^{-1} + C_t r^{-2}$;
			\vspace{2ex}
			\pause
		\item $X_t = X_{f_1(t)}^1 + X_{f_2(t)}^2$ where $X_t^1,X_t^2$ are independent L\'evy processes and $f_1,f_2$ are nondecreasing continuous functions with $f_1(0)=f_2(0)=0$. Then $y_t(r) = f_1(t)h_1(r) + f_2(t)h_2(r)$;
			\vspace{2ex}
			\pause
		\item Semimartingale additive processes with triplets $(B_t,Q_t,\nu_t)$, i.e.\ $B_t$ is of bounded variation,
	\end{itemize}
\end{frame}
\begin{frame}
	\frametitle{Some Other Applications}
	\begin{proposition}[Disintegration of semimartingale additive processes]
		Let $(X_t)_{t\ge0}$ be a semimartingale additive process in $\RR^d$ with triplets $(B_t,Q_t,\nu_t)$. Then there exist a \textcolor{blue}{nondecreasing continuous function} $u$ with $u(0)=0$, a \textcolor{blue}{L\'evy kernel} $k_s(\d x)$, a \textcolor{blue}{vector} $b_s$ and a \textcolor{blue}{nonnegative definite symmetric $d\times d$ matrix} $(q_{ij}(s))$, all of which are locally bounded and left-continuous such that
		\vspace{-2ex}
		$$
		\nu(\d s,\d x) = \kappa_s(\d x)u(\d s),\ B_t=\int_0^t b_s u(\d s),\ (Q_t)_{ij}=\int_0^t q_{ij}(s) u(\d s).
		$$
	\end{proposition}
\end{frame}
\begin{frame}
	\frametitle{Some Other Applications}
	Consequently, we can rewrite our predefined functions $G_t(r),K_t(r)$ and $M_t(r)$ componentwisely to
	\vspace{-1ex}
	\begin{equation*}
		\begin{split}
			& G_t^{(i)}+K_t^{(i)}=\int_0^t\big(r^{-2}\sigma_i(s)^2+\int(x_i/r)^2\wedge1\kappa_s^{(i)}(\d x_i)\big)u(\d s),\\
			& M_t^{(i)}=\left|\int_0^t\tilde m_s^{(i)}(r)u(\d s)\right|,\\
			& \hspace{2cm} \mathrm{\ where\ } \tilde m_s^{(i)}(r)=r^{-1}\left( b_s^{(i)}-\int_{r<|x_i|\le1}x\kappa_s^{(i)}(\d x_i) \right).
		\end{split}
	\end{equation*}
	\pause
	Let 
	$$
	h_s^{(i)}(r) = r^{-2}\sigma_i(s)^2+\int (x_i/r)^2\wedge1\kappa_s^{(i)}(\d x_i)+|\tilde m_s^{(i)}(r)|.
	$$
\end{frame}
\begin{frame}
	\frametitle{Some Other Applications}
	\setlength{\fboxrule}{2pt}
	\fcolorbox{red}{white}{$h_s^{(i)}(r) = r^{-2}\sigma_i(s)^2+\int (x_i/r)^2\wedge1\kappa_s^{(i)}(\d x_i)+|\tilde m_s^{(i)}(r)|$} \\[3ex]
	If, in addition, we have the following two conditions:
	\begin{enumerate}
		\item[(1)] Each $\tilde m_s^{(i)}(r)$ has no sign change in $s$; \\[1ex]
		\item[(2)] $\sum_{i=1}^d h_s^{(i)}(r)$ is nondecreasing in $s$.
	\end{enumerate}
	\vspace{2ex}
	\pause
	Then
	\begin{enumerate}
		\item[(1)] implies $|\int_0^t\tilde m_s^{(i)}(r)u(\d s)|=\int_0^t|\tilde m_s^{(i)}(r)|u(\d s)$ and hence $y_t^{(i)}(r)=\int_0^t h_s^{(i)}(r)u(\d s)$; \\[1ex]
		\item[(2)] implies that for $0<t_1<t_2$ and $v(t)=u^{-1}(t)$,
	\end{enumerate}
\end{frame}
\begin{frame}
	\frametitle{Some Other Applications}
	\small
	\vspace{-4ex}
	\begin{eqnarray*}
		\frac{y_{v(t_1)}(r)}{t_1} &=& t_1^{-1}\int_0^{v(t_1)} \left(\sum_{i=1}^dh_s^{(i)}(r)\right)u(\d s)\\
		&=& t_1^{-1}\int_0^{t_1} \left(\sum_{i=1}^dh_{v(s)}^{(i)}(r)\right)\d s\\
		&\stackrel{(2)}{\le}& t_1^{-1}\int_0^{t_1}\left(\sum_{i=1}^dh_{v( \frac{t_2-t_1}{t_1} s+t_1)}^{(i)}(r)\right)\d s\\ 
		&\le& (t_2-t_1)^{-1}\int_{t_1}^{t_2} \left(\sum_{i=1}^dh_{v(s)}^{(i)}(r)\right)\d s\\
		&=& \frac{y_{v(t_2)}(r)-y_{v(t_1)}(r)}{t_2-t_1},
	\end{eqnarray*}
	from which the quasiconvex condition $y_{v(t_1)}(r)/y_{v(t_2)}(r) \le t_1/t_2$ follows.
\end{frame}
\begin{frame}
	\frametitle{References}
	\tiny
	\begin{thebibliography}{9}
		\bibitem{yang} \textsc{Yang,~M.} (2007). The Growth of Additive Processes. Ann.~Probab.~35 773-805.
		\bibitem{pruitt} \textsc{Pruitt,~W.~E.} (1981). The Growth of Random Walks and L\'evy Processes. Ann.~Probab.~9 948--956.
		\bibitem{le cam} \textsc{Le Cam,~L.} (1986). Asymptotic Methods in Statistical Decision Theory. Springer, New York.
		\bibitem{sato} \textsc{Sato,~K.-I.} (1999). L\'evy Processes and Infinitely Divisible Distributions. Cambridge Univ.~Press, Cambridge.
		\bibitem{ito} \textsc{It\^o,~K.} (2004) Stochastic Processes. Springer, Berlin.
		\bibitem{billingsley} \textsc{Billingsley,~P.} (1995). Probability and Measure, 3rd.~edition. Wiley-Interscience, New York.
		\bibitem{yan} \textsc{He,~S.~W.}, \textsc{Wang,~J.~G.} and \textsc{Yan,~J.~A.} (1992). Semimartingale Theory and Stochastic Calculus. Kexue Chubanshe (Science Press), Beijing.
		\bibitem{shiryaev} \textsc{Jacod,~J.} and \textsc{Shiryaev,~A.~N.} (1987). Limit Theorems for Stochastic Processes. Springer. Berlin.
		\bibitem{de la pena} \textsc{De La Pe\~na,~V.~H.} and \textsc{Eisenbaum,~N.} (1997). Exponential Burkholder Davis Gundy inequalities. Bull.~London Math.~Soc.~29 239-242.
	\end{thebibliography}
\end{frame}
\end{document}


