%        File: DA_Beamer.tex
%     Created: Sun Jun 26 05:00 PM 2011 C
% Last Change: Sun Jun 26 05:00 PM 2011 C
%
\documentclass[12pt]{beamer}
\usefonttheme[stillsansserifmath]{serif}
\usetheme{Madrid}
\defbeamertemplate*{frametitle}{smoothbars theme}
{%
\nointerlineskip%
\begin{beamercolorbox}[wd=\paperwidth,leftskip=.3cm,rightskip=.3cm plus1fil,vmode]{frametitle}
	\vskip.3ex
	\usebeamerfont*{frametitle}\insertframetitle%
	\vskip.3ex
\end{beamercolorbox}%
}
%\let\Tiny=\tiny
\usepackage{lmodern}
\usepackage{amsmath}
\usepackage{amssymb}
\usepackage{amsthm}
\usepackage{mathrsfs}
\usepackage{bbm}
\usepackage{enumitem}
\usepackage{mathtools}
\def\F{\mathcal{F}}
\def\B{\mathcal{B}}
\def\RR{\mathbb{R}}
\def\NN{\mathbb{N}}
\def\i{\mathrm{i}}
\def\e{\mathrm{e}}
\def\d{\mathrm{d}}
\def\k{\mathrm{k}}
\def\Var{\mathrm{Var}}
\def\ind{\mathbf{1}}
\def\la{\langle}
\def\ra{\rangle}
\newcommand{\compl}[1]{{#1}^{c}}
% sigma algebra
\def\sa{$\sigma$- Algebra\ } 
% prob. space
\def\bs{$(\Omega, \mathscr{A}, \mathcal{P})$\ } 
\def\bsigma{\mathscr{B}\brkt{\mathbb{R}^{n}}}
\newcommand{\sqbr}[1]{\left[ {#1} \right]}
\newcommand{\brkt}[1]{\left({#1} \right)}
\newtheorem{proposition}{Proposition}
%%%%
\makeatletter
\newenvironment{customitem}[2]{
  \ifnum\@itemdepth &gt;2\relax\@toodeep\else
  \advance\@itemdepth\@ne%
  \beamer@computepref\@itemdepth%
  \usebeamerfont{itemize/enumerate \beameritemnestingprefix body}%
  \usebeamercolor[fg]{itemize/enumerate \beameritemnestingprefix body}%
  \usebeamertemplate{itemize/enumerate \beameritemnestingprefix body begin}%
  \begin{list}
    {
      \usebeamertemplate{itemize \beameritemnestingprefix item}
    }
    { \leftmargin=#1 \itemindent=#2
      \def\makelabel##1{%
        {%  
          \hss\llap{{%
              \usebeamerfont*{itemize \beameritemnestingprefix item}%
              \usebeamercolor[fg]{itemize \beameritemnestingprefix item}##1}}%
        }%  
      }%  
    }
    \fi
  }
  {
  \end{list}
  \usebeamertemplate{itemize/enumerate \beameritemnestingprefix body end}%
}
\makeatother
%%%%
\title[Diploma Thesis]{Fractional Brownian Motion and its Application in financial mathematics}
\author{Ke Zhu} 
\institute[TU Dresden]{Supervisor: Prof.~Dr.~rer.~nat.~M.~Keller-Ressel\\[2ex] Institute of Mathematical Stochastics\\ TU Dresden}
\date[July 2015]{July ??, 2015}
\begin{document}
\begin{frame}[plain]
	\titlepage
\end{frame}

\begin{frame}
	\frametitle{Contents}
	\begin{itemize}
		\item Introduction of fBm
			\vspace{2ex}
			\pause
		\item Definition and Properties of fBm
			\vspace{2ex}
			\pause
		\item Applications
			\vspace{2ex}
			\pause
			\begin{itemize}
				\item OU-Process 
					\vspace{1.5ex}
					\pause
				\item Fractional Black-Scholes Model
					\vspace{1.5ex}
					\pause
				\item Fractional Volatility Model
			\end{itemize}
	\end{itemize}
\end{frame}
\begin{frame}
	\frametitle{Introduction of fBm}
	Aims
	\begin{itemize}
	 \item Centered Gaussian process $(U_H(t))$ 
	   \vspace{4ex}
	  \item 
	  $\mathrm{Cov}[U_H(t), U_H(s)] = \frac{1}{2} (|t|^{2H} + |s|^{2H} - |t-s|^{2H})$ with Hurst exponent $H\in (0, 1)$.
	  \vspace{2ex}
\end{itemize}
\end{frame}

\begin{frame}
  Mandelbrot and Van Ness
	\frametitle{Integral Representation}
	\begin{definition}
	Let $(U_H(t))_{t\in \mathbb{R}}$ be a $\mathbb{R}$-valued stochatstic process and $H$ be a real number such that $0<H<1$. $(U_H(t))$ is said to be \emph{fractional Brownian motion} if 
  \begin{eqnarray}
		&&U_H(t) - U_H(s)\nonumber\\
		&=& \frac{1}{\Gamma(H+\frac{1}{2})}\brkt{\int_{\mathbb{R}} \mathbbm{1}_{\{t > u\}}\cdot (t-u)^{H-\frac{1}{2}}
		- \mathbbm{1}_{\{s > u\}} (-u)^{H-\frac{1}{2}} \, \mathop{dB_u}}
	\label{sec:frtbm}
  \end{eqnarray}
  for $t\ge s, t, s \in \mathbb{R}$, where $(B_u)$ is defined as two-sides Brownian motion and the integral is defined in the sense of stable integral.
\end{definition}
\end{frame}

\begin{frame}
  \begin{theorem}
	$U_H(0) = 0$, then
	\begin{eqnarray}
	&&U_H(t) \nonumber\\
	&=& \frac{1}{\Gamma(H+\frac{1}{2})}\brkt{\int_{\mathbb{R}} \underbrace{\mathbbm{1}_{\{t > u\}}\cdot (t-u)^{H-\frac{1}{2}} - \mathbbm{1}_{\{u < 0\}} (-u)^{H-\frac{1}{2}}}_{f_t(u)} \, \mathop{dB_u}}.
	\label{sec:fbm}
  \end{eqnarray}
\end{theorem}
\begin{eqnarray*}
  J(f_t) : f_t \rightarrow \int_{\mathbb{R}} f_t \mathop{dB_u}
\end{eqnarray*}
  If the integrand $f$ is quadratic integrable then $(\ref{sec:frtbm})$ is well-defined in the sense of stable integral.
\end{frame}

\begin{frame}
  \begin{theorem}
	 $U_H(t) \sim \mathcal{N}(0, \frac{1}{\Gamma(H+\frac{1}{2})^2}(\int_{\mathbb{R}} |f_t(u)^2|\, \mathop{du}))$
  \end{theorem}

  \begin{proposition}
  The stable integral is linear:
\begin{equation*}
  J(f_t + f_s) \overset{a.s.}{=}J(f_t) + J(f_s)
\end{equation*}
\end{proposition}
  \end{frame}

\begin{frame}
 \begin{theorem}
	$U_H(t)-U_H(s) \sim \mathcal{N}(0, \frac{1}{\Gamma(H+\frac{1}{2})^2}(\int_{\mathbb{R}} |f_t(u)-f_s(u)|^2\, \mathop{du})$
  \end{theorem}
  \begin{theorem}
   The variance of $U_H(t)$ is $\frac{1}{(\Gamma(H+\frac{1}{2}))^2}t^{2H}\, \mathrm{E} U^2_H(1)$ for any $t \in \mathbb{R}$.
  \label{sec:fbmp1}
\end{theorem}
Divide with $\frac{1}{(\Gamma(H+\frac{1}{2}))^2}\mathrm{E} U^2_H(1)$ then, \\
\vspace{2ex}
\hspace{8em}$\mathrm{Var}[U_H(t)]=t^{2H}$.
\end{frame}
\frametitle{Introduction of fBm}
\begin{frame}
  What is a fBm ?
\end{frame}

\frametitle{Introduction of fBm}
\begin{frame}
  \begin{theorem}
  Let  $(U_H(t))_{t}$ be a fBm. The covariance of $U_H(t)$ and $U_H(s)$ is $ \frac{1}{2}(t^{2H} + s^{2H} - |t-s|^{2H})$ for $t, s \in \mathbb{R}$.
\end{theorem}

\begin{proof}
  \begin{eqnarray}
	\mathrm{Cov}[U_H(t), U_H(s)] &=& \mathrm{E}[U_H(t)U_H(s)] \nonumber\\
	&=& \frac{1}{2}(\mathrm{E}[U_H(t)^2] + \mathrm{E}[U_H(s)^2] \nonumber\\
	&-& \mathrm{E}[(U_H(t) - U_H(s))^2]) \nonumber\\
	&=& \frac{1}{2}(t^{2H} + s^{2H} - |t-s|^{2H})
	\label{sec:eqn4}
  \end{eqnarray}
\end{proof}
\end{frame}

\begin{frame}
 \begin{theorem}
  $(U_H(t))_{t}$ is Gaussian process.
\end{theorem}
\begin{theorem}
   Let $(U_H(t))_{t}$ be a fBm, then $(U_H(t))_{t}$ has stationary and H-self similar increments .
\end{theorem} 
\end{frame}

\begin{frame}
  \begin{theorem}
	\begin{eqnarray*}
	   (U_H(t_1),\dots,U_H(t_n)) \sim (U_H(t_1+\tau),\dots,U_H(t_n+\tau))
	\end{eqnarray*}
  \end{theorem}
  \begin{theorem}
\begin{eqnarray*}
  (U_H(ct_1),\dots, U_H(ct_k)) \sim (c^H U_H(t_1),\dots, c^H U_H(t_k))
\end{eqnarray*}
  \end{theorem}
\end{frame}

\begin{frame}
  \begin{definition}
  A stationary stochastic process $(X_t)_t$ is said to have \emph{long memory} if its autocovariance $\varsigma_X(\tau)$ tends to $0$ so slowly such that
  $ \sum_{\tau = 0} ^{\infty} \varsigma_X(\tau)$ diverges.
\end{definition}
\end{frame}

\begin{frame}
  $S_H(k) = U_H(k+1) - U_H(k)$ for $k\in \mathbb{R}$.
  \begin{theorem}
	The fractional Brownian noise $S_H(k)$ with $H \in (\frac{1}{2}, 1)$ has long memory.
  \end{theorem}
\end{frame}
\begin{frame}
  fBm is not semimartingal for $H \neq\frac{1}{2}$.
\end{frame}


\begin{frame}
  \begin{eqnarray*}
 \mathop{dX_t} = -aX_t\mathop{dt} + \gamma \mathop{dU_H(t)},
 \label{sec:ou1}
\end{eqnarray*}
where $(X_t)_{t\ge 0}$ is a stochastic process, $a, \gamma\in\mathbb{R}_{+}$ and $(U_H(t))_{t\ge 0} $ fBm with Hurst exponent $H$. In fact, given an initial condition $X_0(\omega)=b(\omega)$, then in the theory of SDE, (\ref{sec:ou1}) is understood as
\begin{eqnarray*}
  X_t(\omega) = b(\omega) - a\int_0^t X_u(\omega) \mathop{du} + \gamma U_H(t)(\omega)
  \label{sec:oup}
\end{eqnarray*}
for $t \ge 0$.

In order to have a stationary solution, we assume that the initial value is centered Gaussian that $\hat{X}_{H,t}:= \hat{X}_t^{\gamma\int_{-\infty}^0 e^{au}\mathop{dU_H(u)}, H} := e^{-at}\brkt{\gamma\int_{-\infty}^t e^{au}\mathop{dU_H(u)}}$.
\end{frame}

\begin{frame}
  \begin{theorem}
	 $(\hat{X}_{H,t})_{t\ge 0}$ has long memory for $H\in (\frac{1}{2}, 1)$.
  \end{theorem}
\end{frame}

\begin{frame}
  Fractional Black-Scholes model
  \begin{eqnarray*}
  A_t &=& \exp(rt)\nonumber\\
  S_t &=& \exp(rt + \mu(t) +\sigma U_H(t)), t\in [0, T],
  \label{sec:fbs2}
\end{eqnarray*}
where  $r\in\mathbb{R}, \sigma\in\mathbb{R}_+, \sup\limits_{t\in[0, T]}\mu(t) < \infty$.
\end{frame}

\begin{frame}
  \begin{theorem}
  Let $(S_t)_{t\in[0, T]}$ be a stochastic process such that
  \begin{eqnarray}
	\tilde{S}_t = \exp\brkt{\mu(t) + \sigma U_H(t)},
	\label{sec:fbs}
  \end{eqnarray}
  where $\mu, \sigma$ are as in (\ref{sec:fbs2}), $U_H(t)$ is a fBm. If there exist
  \begin{eqnarray*}
  \xi^1_t = f_0\mathbbm{1}_{\{0\}}(t)+\sum_{k=1}^{n-1} f_k \mathbbm{1}_{(\tau_k, \tau_{k+1}]}(t)
  \end{eqnarray*}
  where $t\in[0, T], f_k$ is family of  $\mathcal{F}^{U_H}_k $-measurable function for $k \in \{1,\dots,n-1\}$. $0 = \tau_1 < \cdots <\tau_n = T$ are stopping times with respect to $\mathcal{F}^{U_H}_{\tau_k} $ respectively,  with $\tau_{k+1} - \tau_k\ge m$ for some $m>0$. If there exists a $k \in \{0,\dots,n-1\}$ such that $\mathcal{P}[f_k\neq 0]>0$ , then
  \begin{eqnarray*}
	\mathcal{P}[(\xi^1 \cdot \tilde{S})_T < 0] > 0,
  \end{eqnarray*}
	\label{sec:claim}
	where $(\xi^1 \cdot \tilde{S})_T := \sum_{k=1}^{n} \xi^1_{\tau_k} (\tilde{S}_{k} - \tilde{S}_{k-1})$.
\end{theorem}
\end{frame}

\begin{frame}
   The market (\ref{sec:fbs2}) is arbitrage-free, if there exists a minimal amount of time between two successive transactions,.
\end{frame}

\begin{frame}
\begin{eqnarray}
  \mathop{dS_t} &=& r_tS_t\mathop{dt} + \sigma_t S_t \mathop{dB_t},\\
  \sigma_t &=& \exp\{X_t\} \nonumber\\
  \mathop{dX_t} &=&  -a X_t \mathop{dt} + \gamma \mathop{dU_H(t)},
  \label{sec:fv}
\end{eqnarray}
where $a, \gamma \in \mathbb{R}_{+}$. In the proceeding section, we have a stationary solution 
\begin{eqnarray}
\hat{X}_{H,t}= e^{-at}\gamma\int_{-\infty}^t e^{au}\mathop{dU_H(u)}
\label{sec:fv2}
\end{eqnarray} 
\end{frame}

\begin{frame}
  \begin{theorem}
  $(\hat{\sigma}_{H,t})$ has long memory for $H\in(\frac{1}{2}, 1)$. 
  \end{theorem}
\end{frame}

\begin{frame}
  Smoothness of $\sigma_t$.
  \begin{eqnarray*}
  s(\tau, \sigma) = \frac{1}{N}\sum\limits_{k=1}^N|\log(\sigma_{k\tau}) - \log(\sigma_{(k-1)\tau})|^2,
  \label{sec:smo}
\end{eqnarray*}
\begin{theorem}
  \begin{eqnarray}
	\mathrm{Var}[\hat{X}_{H,t, a}] - \mathrm{Cov}[\hat{X}_{H,t, a}, \hat{X}_{H,t+\tau, a}] \rightarrow \frac{1}{2} \gamma^2\tau^{2H}
	\label{sec:rfsv}
  \end{eqnarray}
 as $a$ goes to zero, for $t>0, \tau>0$.
\end{theorem}
\end{frame}

\begin{frame}
  \begin{definition}
  A \emph{mixed fractional Brownian motion} is defined as follows
\begin{eqnarray}
  M_{\alpha,\beta,H_1,H_2}(t) = \alpha U_{H_1}(t) + \beta U_{H_2}(t)
  \label{sec:mfsv}
\end{eqnarray}
for $t\in \mathbb{R}$, where $\alpha, \beta$ are real numbers and $U_{H_1}, U_{H_2}$ are two independent fBm's with Hurst exponents $H_1 \in (0, \frac{1}{2}), H_2 \in (\frac{1}{2}, 1)$ respectively.
\end{definition}
\end{frame}

\begin{frame}
  \begin{eqnarray*}
	X_{\alpha,\beta,H_1,H_2}(t) = X_{\alpha,\beta,H_1,H_2}(0) - a\int_0^t X_u \mathop{du} + \gamma M_{\alpha,\beta,H_1,H_2}(t).
\end{eqnarray*}

\begin{eqnarray*}
  \hat{X}_{\alpha,\beta,H_1,H_2}(t) &=&  \alpha \gamma e^{-at}\int_{-\infty}^t e^{au} \mathop{d U_{H_1}} + \beta \gamma e^{-at}\int_{-\infty}^t e^{au} \mathop{d U_{H_2}}.
  \label{sec:jjj}
\end{eqnarray*}
\end{frame}

\begin{frame}
  \begin{proposition}
  $\hat{X_t}$ satisfies following properties:
\begin{enumerate}[topsep=0pt, itemsep=-1ex, partopsep=1ex, parsep=1ex, label=(\roman*)]
  \item $(\hat{X_t})_{t\ge 0}$ is a centered Gaussian stationary process.
  \item $\hat{X_t}$ has long memory.
  \end{enumerate}
\end{proposition}
\end{frame}

\begin{frame}
  \begin{proposition}
  Let $M_{\alpha,\beta,H_1,H_2}$ be a weighted fractional brownian motion with respect to $U_{H_1}$ and $U_{H_2}$. $T, \tau>0$, $a, \gamma$ are defined by (\ref{sec:wfsv}). $J_{H_1}, J_{H_2}$ are defined by (\ref{sec:jjj}). $\phi=H_1-\frac{1}{2}, \psi=H_2-\frac{1}{2}$. Then, for $t\in [0, T]$,
\begin{enumerate}[topsep=0pt, itemsep=-1ex, partopsep=1ex, parsep=1ex, label=(\roman*)]	
  \item $\mathrm{E}[\sup\limits_{t\in[0,T]}|\hat{X}_{\alpha,\beta,H_1,H_2}(t) - U_{H_1}(t)|] \rightarrow 0$
	as $a\rightarrow 0, \alpha\rightarrow 1$.
  \item $\mathrm{E}[|\hat{X}_{\alpha,\beta,H_1,H_2}(t+\tau) - \hat{X}_{\alpha,\beta,H_1,H_2}(t)|^2] \rightarrow  \gamma^2 \tau^{2H}$
	as $a\rightarrow 0, \alpha\rightarrow 1$.
  %\item Let $n\in \mathbb{N}$, define
%	\begin{eqnarray*}
%	  &&\tilde{X}_{\alpha,\beta,H_1,H_2}(t) \\
%	  &:=&  \alpha\sum_{k=1}^{\floor{nt}} \frac{(t - \frac{k-1}{n})^\phi}{\Gamma(1+\phi)} \brkt{J_{H_1}^{(\phi)}(\frac{k}{n}) - J_{H_1}^{(\phi)}(\frac{k-1}{n})} \\
%	  &+& \beta\sum_{k=1}^{\floor{nt}} \frac{(t - \frac{k-1}{n})^\psi}{\Gamma(1+\psi)} \brkt{J_{H_2}^{(\psi)}(\frac{k}{n}) - J_{H_2}^{(\psi)}(\frac{k-1}{n})}
%	  \label{sec:disc}
%	\end{eqnarray*}
%	then, as $n$ goes to infinity,
%	\begin{eqnarray*}
%	  \tilde{X}_{\alpha,\beta,H_1,H_2}(t) \rightarrow \hat{X}_{\alpha,\beta,H_1,H_2}(t)
%	\end{eqnarray*}
%	in distribution.
  \end{enumerate}
  \end{proposition}
\end{frame}

%%%%%%%%
%%%%%%%%
%%%%%%%%
\end{document}


